
% -----------------------------------------------------------------------------------
%      PACKAGES & OTHER DOCUMENT CONFIGURATIONS
% -----------------------------------------------------------------------------------
\documentclass[fontsize=11pt]{extarticle}

\usepackage[utf8]{inputenc}
\usepackage[T1]{fontenc}
\usepackage[british]{babel}
% ----------NEW BIBLATEX BIBLIOGRAPHY-----------------------------------------------
\usepackage[eprint=false,backend=bibtex,style = ieee]{biblatex} % Upgrades Bibliography Block Ragged helps break lines in url fixes error

\addbibresource{../BibFile.bib} %%% For biblatex
%e.g to add page number \footfullcite[chapter, p.~215]{AAIB}
% of \footnote{Footnote text goes here}
% This allows can use footfullcite commands
% Note urldate field must be in yyyy-mm-dd to work - use online type
% Remeber to use \printbibliography in the footer
% -----------------------------------------------------------------------------------
% \usepackage{sectsty}
\usepackage{url}

%%% --- The following two lines are what needs to be added --- %%%
\setcounter{biburllcpenalty}{7000}
\setcounter{biburlucpenalty}{8000}

\usepackage{amssymb,amsmath}
\numberwithin{figure}{section} %%%%% <<<<<< Puts Figure Numbering into Sections
\numberwithin{table}{section}%%%%% <<<<<< Puts Tables Numbering into Sections
\usepackage{ifxetex,ifluatex}  %<<<<<<<<< Edit FONT HERE
\ifnum 0\ifxetex 1\fi\ifluatex 1\fi=0 % if pdftex
  \usepackage[T1]{fontenc}
  \usepackage[utf8]{inputenc}
\else % if luatex or xelatex
  \ifxetex
    \usepackage{mathspec}
    \setmainfont[
 BoldFont={AvenirNext-Medium},ItalicFont={AvenirNext-Italic},
 BoldItalicFont={AvenirNext-MediumItalic}]{AvenirNext-Regular}
  \else
  % Font Package for XeLatex
    \usepackage{fontspec}
    \setmainfont{AvenirNext-Regular}
  \fi
  \defaultfontfeatures{Ligatures=TeX,Scale=MatchLowercase}
\fi
\usepackage[fit]{truncate} %Truncates headers that are too long
\usepackage[headheight=26pt,headsep=0.15cm]{geometry}
\usepackage{fancyhdr}
\usepackage{lastpage}
\usepackage{extramarks}
\usepackage{gensymb}
\usepackage{lipsum}
\usepackage{float}
\usepackage{graphicx}
\graphicspath{{../TempImg/}{../Img/}}%<<<<<<<<< Location of Template Images and Other Images, Add folders here
\usepackage{subfig}
\usepackage{wrapfig}
\usepackage[font ={small,it}]{caption}
\usepackage{amsfonts,amsthm} % Math packages
% \usepackage{cite}
\usepackage{csquotes}
%    \MakeAutoQuote{‘}{’}
%    \MakeAutoQuote*{“}{”} %corrects quote marks
\usepackage{enumitem} % resume numbered lists
\usepackage{multicol} %for mulitple colums in lists
\usepackage{booktabs} %<<<<<<<<< Table drawing package
\usepackage[table,xcdraw]{xcolor} %<<<<<<<<< Table drawing package
\usepackage{svg}
\usepackage{scrextend} %call footnotes
\usepackage[colorlinks, linkcolor = black, citecolor = black, filecolor = black, urlcolor = blue]{hyperref} % Creates Hyperlinks for references - add [colorlinks] for coloured hyperlinks
\usepackage{changepage} %Allows Adjust width to be used for the document (indenting paragraphs)
\usepackage{pdfpages} %Allows Pdfpages to be added to the document use \includepdf[pages={1}]{myfile.pdf}
\usepackage{pdflscape} %Change Pages from Portrait to Landscape
\usepackage{color,soul} %% Highlights text for markup
% \usepackage[compact]{titlesec}

\usepackage{pdfpages} %Allows Pdfpages to be added to the document use \includepdf[pages={1}]{myfile.pdf}
%%%%%%%For Condensed Report Format%%%%%%%%%%%%%%
\usepackage{titlesec}
\titlespacing\section{0pt}{6pt plus 2pt minus 2pt}{0pt plus 2pt minus 2pt}
\titlespacing\subsection{0pt}{0pt plus 3pt minus 2pt}{-3pt plus 2pt minus 2pt}
\titlespacing\subsubsection{0pt}{0pt plus 2pt minus 2pt}{-6pt plus 2pt minus 2pt}
\titlespacing\subsubsubsection{0pt}{-6pt plus 2pt minus 2pt}{-6pt plus 2pt minus 2pt}
\setlength{\multicolsep}{-1pt plus 2.0pt minus 1.5pt}% 50% of original values

% \titlespacing*{\section}{0pt}{1.1\baselineskip}{\baselineskip}

\renewcommand*{\thefootnote}{\alph{footnote}} %%% Changes footnotes to letters
\usepackage[bottom]{footmisc} %%% Pushes footnote to bottom and to the margin

\DeclareCiteCommand{\footcite}[\mkbibfootnote]
{\usebibmacro{cite:init}%
\usebibmacro{prenote}}
{\usebibmacro{citeindex}%
\printtext[brackets]{\usebibmacro{cite:comp}}}
{\multicitedelim}
{\usebibmacro{cite:dump}%
\usebibmacro{postnote}}

\newenvironment{indentpara}{\begin{adjustwidth}{2cm}{}}{\end{adjustwidth}} %Declare adjust width wiht indentpara
\renewcommand{\labelitemii}{$\circ$}
\renewcommand{\labelitemiii}{$\diamond$}
\renewcommand{\labelitemiii}{$\cdot$}

% -----------------------------------------------------------------------------------
%                 Code
% -----------------------------------------------------------------------------------
\usepackage{listings}
\lstset{inputpath=Code/}
\usepackage{color}
\definecolor{mygreen}{RGB}{28,172,0} % color values Red, Green, Blue
\definecolor{mylilas}{RGB}{170,55,241}

\lstset{language=Matlab,%
    %basicstyle=\color{red},
    breaklines=true,%
    basicstyle=\small,
    morekeywords={matlab2tikz},
    keywordstyle=\color{blue},%
    morekeywords=[2]{1}, keywordstyle=[2]{\color{black}},
    identifierstyle=\color{black},%
    stringstyle=\color{mylilas},
    commentstyle=\color{mygreen},%
    showstringspaces=false,%without this there will be a symbol in the places where there is a space
    numbers=left,%
    numberstyle={\tiny \color{black}},% size of the numbers
    numbersep=9pt, % this defines how far the numbers are from the text
    emph=[1]{for,end,break},emphstyle=[1]\color{red}, %some words to emphasise
    %emph=[2]{word1,word2}, emphstyle=[2]{style},
}

%% To Add Code Use :
% \lstinputlisting{myfun.m}
%% To input a file or :
% \begin{figure}[h]
% \begin{lstlisting}[language=Matlab]
% \end{lstlisting}
% \catpion{code}
% \end{figure}


% -----------------------------------------------------------------------------------
%                 Quotes
% -----------------------------------------------------------------------------------

\usepackage{epigraph}
% \epigraphsize{\small}% Default
\setlength\epigraphwidth{12cm}
\setlength\epigraphrule{0pt}

\usepackage{etoolbox}
\apptocmd{\sloppy}{\hbadness 10000\relax}{}{}%%%% > Removes Url bibliography warnings
\makeatletter
\patchcmd{\epigraph}{\@epitext{#1}}{\itshape\@epitext{#1}}{}{}
\makeatother

%%%% > For Quotes Use \epigraph{"Quote"}{ - \textup{Author}, Book}

% -----------------------------------------------------------------------------------
%                   NAMES & CLASS DEFINITION %<<<<<<<<< INSERT DETAILS HERE
% -----------------------------------------------------------------------------------

\newcommand{\ModuleTitle}{compgc22 Software Engineering}
\newcommand{\ModuleCode}{COMPGC22}

\newcommand{\AssignmentTitle}{DIY Tool Hire Service}
\newcommand{\University}{University College London}
\newcommand{\Faculty}{Faculty of Engineering Sciences}
\newcommand{\UniCrest}{TempImg/logoucl.png}
\newcommand{\UniLogo}{TempImg/logoucl.png}
\newcommand{\THS}{TempImg/logo.png}

\newcommand{\UniLogoHP}{TempImg/engLogo.jpg}%<<<<<<<<< Make Sure Files are in the Template
\newcommand{\StudentNameA}{Ahmed Afify}

\newcommand{\StudentNameB}{Gideon Acquaah-Harrison}

\newcommand{\StudentNameC}{Daiana Bassi}

\newcommand{\StudentNameD}{Antonio Manganelli}

\newcommand{\StudentNameE}{Cameron Mory}

\newcommand{\StudentNameF}{Stefan Poechhacker}

\newcommand{\StudentNameG}{Caroline Smith}

\newcommand{\StudentNameH}{Phoebe Staab}

\newcommand{\Space}{   }

\newcommand{\SupervisorNameA}{Rae Harbird}
\newcommand{\SupervisorEmailA}{r.harbird@ucl.ac.uk}



% -----------------------------------------------------------------------------------
%        PACKAGES FOR MARKDOWN CONVERSION - FOR USE If Using Markdown to Latex
% -----------------------------------------------------------------------------------
\usepackage{fixltx2e} % provides \textsubscript
% use upquote if available, for straight quotes in verbatim environments
\IfFileExists{upquote.sty}{\usepackage{upquote}}{}
% use microtype if available
\IfFileExists{microtype.sty}{%
\usepackage{microtype}
\UseMicrotypeSet[protrusion]{basicmath} % disable protrusion for tt fonts
}{}
\hypersetup{unicode=true,
            pdftitle={\AssignmentTitle},
            pdfauthor={\AssignmentTitle},
            pdfborder={0 0 0},
            breaklinks=true}
\urlstyle{same}  % don't use monospace font for urls
\usepackage{fancyvrb}
\VerbatimFootnotes % allows verbatim text in footnotes
\usepackage{longtable,booktabs}
\IfFileExists{parskip.sty}{%
\usepackage{parskip}
}{% else
\setlength{\parindent}{0pt}s
\setlength{\parskip}{6pt plus 2pt minus 1pt}
}
\setlength{\emergencystretch}{3em}  % prevent overfull lines
\providecommand{\tightlist}{%
  \setlength{\itemsep}{0pt}\setlength{\parskip}{0pt}}
% \setcounter{secnumdepth}{0}
% Redefines (sub)paragraphs to behave more like sections
\ifx\paragraph\undefined\else
\let\oldparagraph\paragraph
\renewcommand{\paragraph}[1]{\oldparagraph{#1}\mbox{}}
\fi
\ifx\subparagraph\undefined\else
\let\oldsubparagraph\subparagraph
\renewcommand{\subparagraph}[1]{\oldsubparagraph{#1}\mbox{}}
\fi

% -----------------------------------------------------------------------------------
%                   WORD COUTNER - for XeLaTex
% -----------------------------------------------------------------------------------
% \usepackage{xesearch}
% \newcounter{words}
% \newenvironment{counted}{%
%   \setcounter{words}{0}
%   \SearchList!{wordcount}{\stepcounter{words}}
%     {a?,b?,c?,d?,e?,f?,g?,h?,i?,j?,k?,l?,m?,
%     n?,o?,p?,q?,r?,s?,t?,u?,v?,w?,x?,y?,z?}
%   \UndoBoundary{'}
%   \SearchOrder{p;}}{%
%   \StopSearching}

% -----------------------------------------------------------------------------------
%                   MARGINS, HEADERS & FOOTERS
% -----------------------------------------------------------------------------------
 \geometry{
 left=25.4mm,
 right=25.4mm,
 top=25.4mm,
 bottom=25.4mm,
 }
\linespread{1.5}

\pagestyle{fancy}
\lhead{\includegraphics[width = 0.15\textwidth]{\UniLogo}}
% \chead{\AssignmentTitle}
\rhead{}
\lfoot{\AssignmentTitle}
\cfoot{}
\rfoot{Page \thepage} %%%% note the footer is swapped when page numbering style changes
\renewcommand\headrulewidth{0.4pt}
\renewcommand\footrulewidth{0.4pt}

\setlength\parindent{0pt}
 % \setlength{\headheight}{5mm}
\newcommand{\horrule}[1]{\rule{\linewidth}{#1}}

% -----------------------------------------------------------------------------------
%               DOCUMENT STRUCTURE COMMANDS
% -----------------------------------------------------------------------------------
% To sort out the formatting of header and footer when a page...
% ... split occurs "within" a problem environment.
\newcommand{\enterProblemHeader}[1]{
\nobreak\extramarks{#1 (Cont.)}\nobreak
\nobreak\extramarks{#1}{}\nobreak
}
% To sort out the formatting of header and footer when a page...
% ... split occur "between" problem environments.
\newcommand{\exitProblemHeader}[1]{
\nobreak\extramarks{#1 (Cont.)}\nobreak
\nobreak\extramarks{#1}{}\nobreak
}

% -----------------------------------------------------------------------------------
\begin{document}


%For PDF intro
% \includepdf[pages={-}]{DP4front.pdf}

  \setlength{\abovedisplayskip}{-18pt}
  \setlength{\belowdisplayskip}{0pt}
  \setlength{\abovedisplayshortskip}{-18pt}
  \setlength{\belowdisplayshortskip}{0pt}

  \setlist[enumerate]{itemsep=-2mm}
  \setlist[itemize]{itemsep=-2mm}


%----------------------------------------------------------------------------------------
                                  % TITLE PAGE FORMAT
%----------------------------------------------------------------------------------------
\pagenumbering{roman}
\begin{titlepage}

  \center % Center everything on the page
%----------------------------------------------------------------------------------------
% HEADING SECTION
%----------------------------------------------------------------------------------------
    \usefont{OT1}{bch}{b}{n}
    \normalfont \normalsize \textsc{\University} \\ [10pt]
    \normalfont \normalsize \textsc{\Faculty} \\ [25pt]
    
%----------------------------------------------------------------------------------------
% LOGO SECTION - Adds Univeristy Crest to the Report
%----------------------------------------------------------------------------------------
    % \includegraphics[width = 0.5\textwidth]{\UniLogoHP}\\[0.5cm]
    \includegraphics[width = 0.4\textwidth]{\UniCrest}\\[0.5cm]
    
%----------------------------------------------------------------------------------------
% HEADING SECTION
%----------------------------------------------------------------------------------------
    \normalfont \normalsize \textsc{\ModuleTitle} \\ [10pt]
%----------------------------------------------------------------------------------------
% TITLE SECTION
%----------------------------------------------------------------------------------------
    \horrule{0.5pt} \\[0.4cm]
    \includegraphics[width = 0.85\textwidth]{\THS}\\[0.5cm]
    % \huge \textbf{\AssignmentTitle} \\
    \horrule{2pt} \\[0.5cm]

%----------------------------------------------------------------------------------------
% AUTHOR SECTION
%----------------------------------------------------------------------------------------
\begin{minipage}{0.4\textwidth}
  \begin{flushleft} \large
  \emph{Supervisor:}\\
  % Change Name
  \textbf{\SupervisorNameA}\\
  % \textbf{\SupervisorNameB}
  \end{flushleft}
\end{minipage}\qquad
\begin{minipage}{0.4\textwidth}
  \begin{flushright} \large
  \emph{Email:} \\
  \SupervisorEmailA \\
  % \SupervisorEmailB
  \end{flushright}
\end{minipage} \\[1cm]
\begin{minipage}{0.4\textwidth}
  \begin{flushleft} \large
  \emph{Authors:} \\
    \textbf{\StudentNameA} \\
    \textbf{\StudentNameB} \\
    \textbf{\StudentNameC} \\
    \textbf{\StudentNameD}
    \textbf{\Space}
  \end{flushleft}
\end{minipage}\qquad
\begin{minipage}{0.4\textwidth}
  \begin{flushright} \large
  \emph{ } \\
    \textbf{\StudentNameE} \\
    \textbf{\StudentNameF} \\
    \textbf{\StudentNameG} \\
    \textbf{\StudentNameH}
    \textbf{\Space}
  \end{flushleft}
\end{minipage}\qquad

  \\[0.5cm]


%----------------------------------------------------------------------------------------
% DATE SECTION
%----------------------------------------------------------------------------------------

\textit{{\large \today}}\\% Date, change the \today to a set date if you want to be precise

%----------------------------------------------------------------------------------------
% Disclaimer
%----------------------------------------------------------------------------------------
{\footnotesize This report is submitted as part requirement for the COMPGC22 module at UCL. It is substantially the result of our own work except where explicitly indicated in the text. The report may be freely copied and distributed provided the source is explicitly acknowledged.}

% ---------------------------------
\vfill % Fill the rest of the page with whitespace
\end{titlepage}

% \setcounter{page}{3}

\newpage


% -----------------------------------------------------------------------------------
%                                ABSTRACT
% -----------------------------------------------------------------------------------
\renewcommand{\abstractname}{Executive Summary}
\addcontentsline{toc}{section}{Executive Summary}
\begin{abstract}

The key objective of this body of work is to illustrate the process by which Group A designed the core functionalities of the web-based DIY Tool Hire Service (THS) application. During the ideation stage of the project, the team identified the scope of the project and the core requirements of the DIY THS. Through the use of a SCRUM embedded Unified Process workflow, the team developed a blue print for the application from requirements and use case evaluation through to object-oriented analysis and prototype design. These achievements were possible through the use of the Unified Modelling Language (UML) as the medium through which information was shared. 

Keeping with the tenants of the Unified Process, requirements and use cases were at the core of all system designs. As such, the development of the key requirements of the system and the use cases through which they are performed were developed through a rigorous and iterative process over several weeks. The results of this effort can be seen and explored further in Chapter 1. These served as a guide for the subsequent object-oriented analysis and architecture design. 

During the object-oriented analysis of the DIY THS, the team kept with the aforementioned iterative approach. After the development of plain-English activity diagrams for key use-cases, the team endeavored to create some basic pseudo code to carry out such processes. This lead to the development of class diagrams which were, again, use-case driven. The development of the design class diagram and sequence diagrams were closely linked and iterated over for several weeks to ensure consistency between the models. The design class diagram and the objects contained within it were predecessors to the development of state-machine diagrams.

The system architecture was explored through the use of further UML models such as component and deployment diagrams. The DIY THS would be implemented on a three-tier architectural model which is discussed in more detail along with the architectural UML models in Chapter 4. 

Finally, a mock-up of prototype designs is presented at the close of this work to bring the UML models to life. These illustrations also help to emphasize the particulars of the role of the presentation tier in the three-tier system. Alongside the figures in Chapter 5 are detailed descriptions of the use-cases related to their respective mock-up designs. 

While future iterations of this work would not only be possible, but encouraged before implementation, this report ultimately gives a realistic and pragmatic proposed approach to the development of a web-based tool hire service. 


\end{abstract}
\newpage

% -----------------------------------------------------------------------------------
%                              TABLE OF CONTENTS
% -----------------------------------------------------------------------------------

\tableofcontents
%
%
\newpage

\addcontentsline{toc}{section}{List of Tables}
\listoftables
\addcontentsline{toc}{section}{List of Figures}
\listoffigures

\newpage

\addcontentsline{toc}{section}{List of Acronyms}
\section*{List of Acronyms}\label{acronyms}
\textbf{THS}: Tool Hire Service \\
\textbf{DIY}: Do it yourself \\
\textbf{USDP}: Unified Software Development Process  \\
\textbf{MVP}: Minimum viable product \\
\textbf{OO}: Object oriented \\
\textbf{UP}: Unified Process \\
\textbf{UC}: Use case \\
\textbf{UML}: Unified Modelling Language\\



\newpage

%% -----------------------------------------------------------------------------------
%%                              INTRODUCTION
%% -----------------------------------------------------------------------------------
\clearpage
\rfoot{Page \thepage}
\pagenumbering{arabic}
% \begin{counted} %<<<<<<<<<<<<<<STARTS WORD COUNTER
